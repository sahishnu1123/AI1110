\documentclass{beamer}
\usetheme{CambridgeUS}

\title{Assignment 8}
\author{Bhogalapalli Sahishnu, CS21BTECH11009}
\date{June 13, 2022}
\begin{document}

\begin{frame}
    \titlepage
\end{frame}

\section{Question}
\begin{frame}{Question}
\textbf{Papoulis-Pillai Ch 9 Ex 9-21: }\\
The process $x(t)$ is normal with zero mean. Show that if $y(t) = x^2(t)$, then
\begin{align*}
    S_y(w) = 2\pi {R_{x}}^{2}(0) \delta(\omega) + 2 S_{x}(\omega) * S_x(\omega)
\end{align*}
\end{frame}

\section{Theory}
\begin{frame}{Theory}
\textbf{Frequency Convolution Theorem}: \\
The frequency convolution theorem states that the multiplication of two signals in time domain is equivalent to the convolution of their spectra in the frequency domain.\\
We also know that moment generating function of x and y is given by,
\begin{align}
    \phi(s_1,s_2) &= E\{e^{s_1x + s_2y}\}\\
                  &= \sum_{n=0}^{\infty} \frac{1}{n!} \sum_{n=0}^{n} {n \choose k} E\{x^k y^{n-k}\} {s_1}^k {s_2}^{n-k}\\
                  &= 1 + m_{10}s_1 + m_{01}s_2 + \frac{1}{2}(m_{20}{s_1}^2 + 2m_{11}s_2 + m_{02}{s_2}^2)
\end{align}
\end{frame}

\section{Solution}
\begin{frame}{Solution}
We can see that,
\begin{align*}
    R_{y}(\tau) = E\{x^2(t+\tau)x^2(t)\}
\end{align*}
From equation 3,
\begin{align*}
    R_y(\tau) &= E\{x^2(t + \tau)\} E\{x^2(t)\} + 2 E^2\{x(t+\tau) x(t)\}\\
              &= {R_{x}}^2(0) + 2{R_{x}}^2(\tau)
\end{align*}
From above conclusion and frequency convolution theorem, it follows that
\begin{align}
    S_y(w) = 2\pi {R_x}^2(0)\delta(\omega) + \frac{1}{\pi} S_x(\omega) * S_x(\omega)
\end{align}

    
\end{frame}

\end{document}
